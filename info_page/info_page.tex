\documentclass{article}
\usepackage{graphicx} % Required for inserting images

%-------------------------------
%           HYPERLINKS
%-------------------------------

\usepackage{hyperref}

%-------------------------------
%           MARGINS
%--------------------------------

\usepackage[top=2cm]{geometry}

%-------------------------------
%            HEBREW
%-------------------------------

\usepackage{hebrewdoc}

%-------------------------------
%            LISTS
%-------------------------------

\usepackage{enumitem}

%-------------------------------
%            TITLE
%-------------------------------

\title{דף מידע \\ אלגברה ב' - 01040168 \\ אביב 2025}
\date{}

\begin{document}

\maketitle

\section*{סגל הקורס}

\begin{description}
\item[מרצה:]
אריאל רפפורט
-
אמדו 915
-
\textenglish{\href{mailto:arapaport@technion.ac.il}{arapaport@technion.ac.il}}
\\
\emph{משרד:}
אמאדו 915.
\\
\emph{שעת קבלה:}
יום ג' 12:30-13:30 במשרד, או בתיאום מראש.

\item[מתרגל אחראי:] 
יואב סנדנר
-
\textenglish{\href{mailto:sandneryoav@campus.technion.ac.il}{sandneryoav@campus.technion.ac.il}}
\\
\emph{שעת קבלה:}
יום ב' 15:30-16:20 באמאדו 617, או בתיאום מראש.

\item[מתרגל:] 
אלן סורני
-
\textenglish{\href{mailto:elad.tzorani@campus.technion.ac.il}{elad.tzorani@campus.technion.ac.il}}
\\
\emph{שעת קבלה:}
יום ד' 16:30-17:20 באולמן 505, או בתיאום מראש.

\end{description}

\section*{מבנה הציון}

הציון הסופי בקורס יקבע באופן הבא.

\begin{itemize}
\item[-] \emph{10\%}
מגן עבור שיעורי בית.
\\
רכיב זה יקבע לפי ממוצע $n-2$ גיליונות שיעורי הבית הטובים ביותר מתוך $n$ גיליונות. הגיליונות יהיו שבועיים והציון עבור כל גיליון יהיה מתוך 3 ויקבע לפי מספר הפתרונות שהוגשו.
\\
המגן ילקח בחשבון בתנאי שהוא גבוה מציון המבחן, ובתנאי שהציון במבחן הינו עובר.
\item[-] \emph{90-100\%}
בחינה סופית.
\end{itemize}

\section*{מידע כללי}

\begin{itemize}
\item[-]
יפורסם במהלך הקורס בוחן דמה לתרגול עצמי, ללא בדיקה או ציון.
\item[-]
פניות בנושאי ניהול הקורס ושיעורי בית יש להפנות ליואב המתרגל האחראי.
\item[-]
מותר להגיש תרגילי בית עד 5 ימי איחור מצטברים בלי אישור מיוחד. הארכות נוספות ינתנו עקב מילואים או מקרים רפואיים בלבד, באישור בכתב ובהתאם לשיקול הדעת של סגל הקורס.
\item[-]
יש להגיש פתרון להגשה בקובץ \textenglish{.pdf} יחיד בלבד, עם עמודים בגודל סטנדרטי, בכתב קריא, ועם דפים מיושרים כלפי מעלה.
\\
מותר ואף רצוי להגיש פתרונות מוקלדים.
\end{itemize}

\pagebreak

\section*{סילבוס}
\begin{enumerate}
\item הדטרמיננטה: הגדרה ותכונות. מטריצה מצורפת וכלל קרמר.
\item סכומים ישרים, הטלות, תת־מרחבים שמורים וצמצום של אופרטור לינארי.
\item אופרטורים נילפוטנטיים ואינדקס הנילפוטנטיות. צורת ז'ורדן.
\item משפט קיילי המילטון ומשפט הפירוק הפרימרי.
\item מרחבי מכפלה פנימית, המשלים הניצב ותהליך גרם־שמידט. הטלות אורתוגונליות.
\item האופרטור הצמוד, אופרטורים נורמליים, אופרטורים צמודים לעצמם, אופרטורים אוניטריים ואיזומטריות.
\item משפט הפירוק הספקטרלי ולכסון אורתוגונלי. אופרטורים חיוביים ומשפט הפירוק הפולרי. \textenglish{SVD}.
\item תבניות בילינאריות ותבניות ריבועיות. חפיפת מטריצות. משפט סילבסטר.
\item אלגברה מולטילינארית ומכפלות טנזוריות.
\end{enumerate}

\section*{ספרות}
\begin{itemize}
\item[-] סיכומי הרצאות של ניר לזרוביץ' (נמצאים במודל)
\item[-] \textenglish{Algebra (2nd Edition) - M.~Artin}
\item[-] \textenglish{Linear Algebra - K.~Hoffman and R.F.~Kunze}
\item[-] \textenglish{Linear Algebra Done Right - S.~Axler}
\end{itemize}

\end{document}