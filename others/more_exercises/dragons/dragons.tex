\documentclass[a4paper,10pt,twoside,openany]{article}

\usepackage[lang=hebrew]{maths}
\usepackage{hebrewdoc}
\usepackage{stylish}
\usepackage{lipsum}
\let\bs\blacksquare

\setlength{\parindent}{0pt}

%%%%%%%%%%%%
% Styling %
%%%%%%%%%%%%

\usepackage{enumitem}

%%%%%%%%%%%%%
% Counters  %
%%%%%%%%%%%%%

\setcounter{section}{0}     
            
%BIBLIOGRAPHY
\usepackage[
backend=biber,
style=alphabetic,
]{biblatex}
\addbibresource{bibliography.bib} %Imports bibliography file

%%%%%%%%%%
% Title  %
%%%%%%%%%%
\title{
אלגברה ב' - תרגיל על צורת ז'ורדן
\\
\vspace{1cm}
\large{לא להגשה}
}
\date{}

\begin{document}
\maketitle

\begin{exercise}
בתרגיל זה נראה כיצד צורת ז'ורדן עוזרת בחישוב בעיות המצריכות חזקות של מטריצות.

\begin{enumerate}
\item
תהיינה
$A,B,P \in \Mat_n\prs{\mbb{C}}$
כאשר
$P$
הפיכה וגם
$A = P^{-1} B P$.
הראו כי
$A^r = P^{-1} B^r P$
לכל
$r \in \mbb{N}$.

\item
בשמורת הטבע ליד הטכניון סין יש היום 2 דרקונים, 600 פנדות ו־20000 במבוקים.

כל שנה הדרקונים, הפנדות והבמבוקים מתרבים ומספרם גדל פי 2.

לאחר מכן, כל פנדה אוכלת במבוק אחד וכל דרקון אוכל שתי פנדות.

אז, רשות הטבע והגנים הסינית משחררת לטבע 4 דרקונים ו־10 פנדות, אם עדיין יש פנדות בשמורה.

לבסוף, אם לא נשאר במבוק בסוף השנה, כל הפנדות מתות.

\begin{enumerate}
\item
מיצאו מטריצה
$A \in \Mat_4\prs{\mbb{C}}$
וערכים
$d,p,b$
עבורם מספרי הדרקונים, הפנדות והבמבוקים בסוף השנה ה־%
$t$
הם מקדמים בוקטור
$A^t \pmat{1 \\ d \\ p \\ b}$
לכל
$t \in \mbb{N} \cup \set{0}$.

\item נשיא הטכניון מתכנן לבקר בסין עוד 30 שנה. האם יהיו פנדות בשמורה בזמן הביקור שלו?

\item הטכניון החליט להעביר את הלימודים מסין למאדים עוד 230 שנה. האם ישארו עד אז פנדות בשמורת הטבע?
\end{enumerate}
\end{enumerate}
\end{exercise}

\end{document}