\documentclass[a4paper,10pt,twoside,openany]{article}

\usepackage[lang=hebrew]{maths}
\usepackage{hebrewdoc}
\usepackage{stylish}
\usepackage{lipsum}
\let\bs\blacksquare

\setlength{\parindent}{0pt}

%%%%%%%%%%%%
% Styling %
%%%%%%%%%%%%

\usepackage{enumitem}

%%%%%%%%%%%%%
% Counters  %
%%%%%%%%%%%%%

\setcounter{section}{1}     
            
%BIBLIOGRAPHY
\usepackage[
backend=biber,
style=alphabetic,
]{biblatex}
\addbibresource{bibliography.bib} %Imports bibliography file

%%%%%%%%%%
% Title  %
%%%%%%%%%%
\title{
אלגברה ב' - הצעה לפתרון מועד א'
}
\date{}

\begin{document}
\maketitle

\setcounter{exercise}{4}

\begin{exercise}
יהי
$W$
התת־מרחב הלינארי של
$\mbb{R}^4$
אשר נפרש על ידי הוקטורים
$\pmat{1 \\ 1 \\ 0 \\ 0}$
ו־%
$\pmat{1 \\ 1 \\ 1 \\ 2}$.
מצאו את הוקטור
$w \in W$
כך ש־%
$\norm{w - \pmat{1 \\ 2 \\ 3 \\ 4}}$
קטן כלל האפשר, כאשר
$\norm{\cdot}$
היא הנורמה המושרת מהמכפלה הפנימית הסטנדרטית של
$\mbb{R}^4$.
\end{exercise}

\begin{solution}
אנו מחפשות את הוקטור ב־%
$W$
שקרוב ביותר לוקטור
$x \coloneqq \pmat{1 \\ 2 \\ 3 \\ 4}$.
לפי משפט, וקטור זה שווה להטלה האורתוגונלית $P_W\prs{u}$ של
$u$
על
$W$.
לשם חישוב ההטלה האורתוגונלית, נמצא בסיס אורתונורמלי של
$W$,
בעזרת תהליך גרם־שמידט על הבסיס
$B \coloneqq \prs{u_1, u_2} = \prs{\pmat{1 \\ 1 \\ 0 \\ 0}, \pmat{1 \\ 1 \\ 1 \\ 2}}$
של
$W$.

ננרמל את הוקטור הראשון בבסיס,
\[\text{.} v_1 = \frac{u_1}{\norm{u_1}} = \frac{u_1}{\sqrt{1^2 + 1^2}} = \frac{1}{\sqrt{2}} \pmat{1 \\ 1 \\ 0 \\ 0}\]
נחסר מהוקטור השני את ההטלה שלו על המרחב הנפרש על ידי
$v_1$.
\begin{align*}
w_2 &= u_2 - \trs{u_2, v_1} v_1
\\&=
u_2 - \trs{u_2, \frac{1}{\sqrt{2} u_1}} \cdot \frac{1}{\sqrt{2}} u_1
\\&=
\pmat{1 \\ 1 \\ 1 \\ 2} - \frac{1}{2} \trs{\pmat{1 \\ 1 \\ 0 \\ 0}, \pmat{1 \\ 1 \\ 1 \\ 2}} \pmat{1 \\ 1 \\ 0 \\ 0}
\\&=
\pmat{1 \\ 1 \\ 1 \\ 2} - \frac{1}{2} \cdot \prs{1 + 1} \pmat{1 \\ 1 \\ 0 \\ 0}
\\&=
\pmat{0 \\ 0 \\ 1 \\ 2}
\end{align*}
ננרמל את הוקטור שקיבלנו.
\begin{align*}
v_2 = \frac{w_2}{\norm{w_2}} = \frac{w_2}{\sqrt{1^2 + 2^2}} = \frac{1}{\sqrt{5}} \pmat{0 \\ 0 \\ 1 \\ 2}
\end{align*}

\newpage

כעת, לפי נוסחה עבור הטלה אורתוגונלית, ההטלה האורתוגונלית על תת־מרחב
$U \leq V$
עם בסיס אורתונורמלי
$\prs{e_1, \ldots, e_m}$
היא
\[\text{.} P_U\prs{v} = \sum_{i \in \brs{m}} \trs{v, e_i} e_i\]
אצלנו נקבל כי
\begin{align*}
P_W\prs{x} &= \trs{x, v_1} v_1 + \trs{x, v_2} v_2
\\&=
\trs{\pmat{1 \\ 2 \\ 3 \\ 4}, \frac{1}{\sqrt{2}} \pmat{1 \\ 1 \\ 0 \\ 0}} \cdot \frac{1}{\sqrt{2}} \pmat{1 \\ 1 \\ 0 \\ 0} + \trs{\pmat{1 \\ 2 \\ 3 \\ 4}, \frac{1}{\sqrt{5}} \pmat{0 \\ 0 \\ 1 \\ 2}} \cdot \frac{1}{\sqrt{5}} \pmat{0 \\ 0 \\ 1 \\ 2}
\\&= \frac{1}{2} \trs{\pmat{1 \\ 2 \\ 3 \\ 4}, \pmat{1 \\ 1 \\ 0 \\ 0}} \pmat{1 \\ 1 \\ 0 \\ 0} + \frac{1}{5} \trs{\pmat{1 \\ 2 \\ 3 \\ 4}, \pmat{0 \\ 0 \\ 1 \\ 2}} \pmat{0 \\ 0 \\ 1 \\ 2}
\\&= \frac{3}{2} \pmat{1 \\ 1 \\ 0 \\ 0} + \frac{11}{5} \pmat{0 \\ 0 \\ 1 \\ 2}
\\ \text{,} \hphantom{P_W\prs{x}} &= \pmat{\frac{3}{2} \\ \frac{3}{2} \\ \frac{11}{5} \\ \frac{22}{5}}
\end{align*} 
וכאמור, זה הוקטור הקרוב ביותר ל־%
$x$
ב־%
$W$,
כנדרש.
\end{solution}

\setcounter{exercise}{7}

\newpage

\begin{exercise}
האם קיים פולינום
$p\prs{x} \in \mbb{R}\brs{x}$
ממעלה לכל היותר
$5$
כך שלכל פולינום
$q\prs{x} \in \mbb{R}\brs{x}$
ממעלה לכל היותר
$5$
מתקיים
$q'\prs{2} = \int_3^4 q\prs{t} p\prs{t} \diff t$?
\end{exercise}

\begin{solution}
כן.

ראינו בכיתה כי
\[\trs{q, p} \coloneqq \int_3^4 q\prs{t} p\prs{t} \diff t\]
הינה מכפלה פנימית על
$\mbb{R}_{\geq 5}\brs{x}$.
נראה כי
\begin{align*}
\phi \colon \mbb{R}_{\geq 5}\brs{x} &\to \mbb{R} \\
q &\mapsto q'\prs{2}
\end{align*}
הינו פונקציונל לינארי. לפי משפט ריס, אם
$\psi$
פונקציונל לינארי על מרחב מכפלה פנימית
$V$,
קיים
$w \in V$
עבורו
$\phi\prs{v} = \trs{v, w}$
לכל
$v \in V$.
אצלנו נקבל כי קיים
$p \in \mbb{R}_{\geq 5}\brs{x}$
עבורו
\[\text{,} q'\prs{2} = \phi\prs{q} = \trs{q, p} = \int_2^3 q\prs{t} p\prs{t} \diff t\]
כנדרש.

אכן,
$\phi$
פונקציונל לינארי, כי עבור
$q_1, q_2 \in \mbb{R}_{\geq 5}\brs{x}$
ועבור
$\alpha \in \mbb{R}$
מתקיים כי
\begin{align*}
\phi\prs{\alpha q_1 + q_2} &= \prs{\alpha q_1 + q_2}' \prs{2}
\\&= \prs{\alpha q_1' + q_2'} \prs{2}
\\&= \alpha q_1'\prs{2} + q_2'\prs{2}
\\&= \alpha \phi\prs{q_1} + \phi\prs{q_2}
\end{align*}
כאשר בשוויון השני השתמשנו בלינאריות הנגזרת ובשוויון השלישי בהגדרת סכום וכפל בסקלר של פונקציות.
\end{solution}

\newpage

\begin{exercise}
יהי
$V$
מרחב מכפלה פנימית סוף־מימדי מעל
$\mbb{C}$,
ותהי
$T \in \End_{\mbb{C}}\prs{V}$
איזומטריה. האם בהכרח קיימת איזומטריה
$S \in \End_{\mbb{C}}\prs{V}$
כך ש־%
$S^2 = T$?
\end{exercise}

\begin{solution}
כן.

ניזכר כי
$T$
איזומטריה אם
$\norm{Tv} = \norm{v}$
לכל
$v \in V$,
וכי ראינו בהרצאה שזה שקול לכך ש־%
$T^* T = \id_V$,
כלומר לכך ש־%
$T$
אוניטרית.
בפרט נקבל כי
$T$
נורמלית, ולכן לפי משפט הפירוק הספקטרלי קיים בסיס אורתונורמלי
$B$
של
$V$
וקיימות
$\lambda_1, \ldots, \lambda_n$
עבורם
$\brs{T}_B = \diag\prs{\lambda_1, \ldots, \lambda_n}$
מטריצה אלכסונית.

ניזכר כי אופרטור נורמלי הינו אוניטרי אם ורק אם כל הערכים העצמיים שלו על מעגל היחידה. לכן
$\abs{\lambda_i} = 1$
לכל
$i \in \brs{n}$,
ונוכל לכתוב
$\lambda_i = \mrm{cis}\prs{\theta_i}$
כאשר
$\mrm{cis}\prs{\theta} \coloneqq \cos\prs{\theta} + i \sin\prs{\theta}$
ועבור ערכים
$\theta_i \in \mbb{R}$.

נסמן
$\mu_i \coloneqq \mrm{cis}\prs{\frac{\theta_i}{2}}$
לכל
$i \in \brs{n}$
ונקבל כי
\begin{align*}
\mu_i^2 &=
\prs{\mrm{cis}\prs{\frac{\theta_i}{2}}}^2
\\&= \mrm{cis}\prs{2 \cdot \frac{\theta_i}{2}}
\\&= \mrm{cis}\prs{\theta_i}
\\ \text{.} \hphantom{\mu_i^2} &= \lambda_i
\end{align*}
יהי
$S \in \End_{\mbb{C}}\prs{V}$
האופרטור עבורו
$\brs{S}_B = \diag\prs{\mu_1, \ldots, \mu_n}$.

אז
$S$
נורמלי לפי משפט הפירוק הספקטרלי כי קיים בסיס אורתונורמלי
$B$
המלכסן את
$S$,
ו־%
$S$
אוניטרי כי הוא נורמלי עם ערכים עצמיים על מעגל היחידה, ולפי אותה טענה מהכיתה בה השתמשנו קודם.
בנוסף,
\begin{align*}
\brs{S^2}_B &= \brs{S}_B^2
\\&= \diag\prs{\mu_1, \ldots, \mu_n}^2
\\&= \diag\prs{\mu_1^2, \ldots, \mu_n^2}
\\&= \diag\prs{\lambda_1, \ldots, \lambda_n}
\\&= \brs{T}_B
\end{align*}
ולכן
$S^2 = T$,
כנדרש.
\end{solution}

\newpage

\begin{exercise}
האם קיימות מטריצות
$A_1, \ldots, A_6 \in M_2\prs{\mbb{C}}$
כך ש־%
$A_i$
ו־%
$A_j$
אינן דומות לכל
$1 \leq i < j \leq 6$
וכן הפולינום האופייני של
$A_i^2$
שווה ל־%
$x^2 - 2x + 1$
לכל
$1 \leq i \leq 6$?
\end{exercise}

\begin{solution}
לא.

עבור מטריצה ריבועית
$A$,
הערכים העצמיים של
$A^2$
הם ריבועי הערכים העצמיים של
$A$.
הערכים העצמיים של מטריצה הם שורשי הפולינום האופייני שלה, ולכן אם
$A^2$
עם פולינום אופייני
$x^2 -2x + 1 = \prs{x-1}^2$,
הערך העצמי היחיד של
$A^2$
הוא
$1$
ואז הערכים העצמיים האפשריים של
$A$
הם ערכי
$\lambda \in \mbb{C}$
עבורם
$\lambda^2 = 1$,
כלומר
$\pm 1$.

נניח בשלילה שקיימות
$A_1, \ldots, A_6$
כמתואר בשאלה, ונקבל כי לכולן ערכים עצמיים בקבוצה
$\set{1, -1}$.

אם ל־%
$A \in M_2\prs{\mbb{C}}$
ערכים עצמיים בקבוצה
$\set{1, -1}$,
צורת ז'ורדן שלה הינה בהכרח אחת מבין $5$ המטריצות הבאות.

\begin{align*}
J_2\prs{1}, J_2\prs{-1}, \diag\prs{1, 1}, \diag\prs{1, -1}, \diag\prs{-1, -1}
\end{align*}

מעקרון שובך היונים נקבל כי קיימים
$1 \leq i < j \leq 6$
עבורם ל־%
$A_i, A_j$
אותה צורת ז'ורדן
$J$.
אבל אז
\[A_i \cong J \cong A_j\]
ולכן
$A_i$
ו־%
$A_j$
דומות, בסתירה להנחה.
\end{solution}

\end{document}