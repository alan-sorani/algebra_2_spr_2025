\documentclass[a4paper,10pt,twoside,openany]{article}

\usepackage[lang=hebrew]{maths}
\usepackage{hebrewdoc}
\usepackage{stylish}
\usepackage{lipsum}
\let\bs\blacksquare

\setlength{\parindent}{0pt}

%%%%%%%%%%%%
% Styling %
%%%%%%%%%%%%

\usepackage{enumitem}

%%%%%%%%%%%%%
% Counters  %
%%%%%%%%%%%%%

\setcounter{section}{0}     
            
%BIBLIOGRAPHY
\usepackage[
backend=biber,
style=alphabetic,
]{biblatex}
\addbibresource{bibliography.bib} %Imports bibliography file

%%%%%%%%%%
% Title  %
%%%%%%%%%%
\title{
אלגברה ב' - טענה בנוגע לדרך מציאת בסיס ז'ורדן
}
\date{}

\begin{document}
\maketitle

\begin{propositionstarred}
יהי
$V$
מרחב וקטורי ממימד סופי $n$ ומעל שדה
$\mbb{F}$,
יהי
$T \in \End_{\mbb{F}}\prs{V}$
אופרטור נילפוטנטי עם אינדקס נילפוטנטיות
$k$,
ולכל
$i \in \brs{k}$
יהי
$B_i \coloneqq \prs{v_{i,1}, \ldots, v_{i,n_i}}$
בסיס עבור
$\ker\prs{T^i} \setminus \ker\prs{T^{i-1}}$.
לכל
$v \in \ker\prs{T^\ell} \setminus \ker\prs{T^{\ell - 1}}$
נסמן
$C_v = \prs{T^{\ell-1}\prs{v}, \ldots, T\prs{v}, v}$
וכן
$V_{v_{i,j}} = \Span\prs{C_v}$.

אז ניתן לקבל בסיס ז'ורדן עבור
$T$
באופן הבא.

\begin{enumerate}
\item
ניקח
$B = \biguplus_{j \in n_k} C_{v_{k,j}}$
וניקח
$i = k$.

\item
נקטין את
$i$
באחת.
נבחר תת־קבוצה סדורה מקסימלית
$\tilde{B}_i$
של
$B_i$
שהינה בלתי־תלויה לינארית בוקטורים ב־%
$B$
(היא תהיה מגודל
$2\dim \ker\prs{T^i} - \dim\ker\prs{T^{i+1}} - \dim\ker\prs{T^{i-1}}$),
ולכל וקטור
$v \in \tilde{B}_i$
נשרשר ל־%
$B$
את
$C_v$.
כלומר נעדכן
\[\text{.} B_{\text{new}} = B_{\text{old}} \uplus C_v\]

\item
אם
$B$
באורך
$n$,
נסיים. אחרת נחזור לשלב הקודם.
\end{enumerate}

\end{propositionstarred}

לשם הוכחת הטענה, נוכיח תחילה שתי למות לגבי סכומים ישרים.

\begin{lemma} \label{lemma:basis-projection}
יהי
$V$
מרחב וקטורי סוף־מימדי מעל שדה
$\mbb{F}$,
יהי
$T \in \End_{\mbb{F}}\prs{V}$
ויהיו
$V_1, V_2 \leq V$
תת־מרחבים עבורם
$V = V_1 \oplus V_2$.
תהי
$P$
ההטלה על
$V_1$
במקביל ל־%
$V_2$,
כלומר ההעתקה המקיימת
$P\prs{v_1 + v_2} = v_1$
לכל
$v_1 \in V_1$
ולכל
$v_2 \in V_2$.

יהי
$B = \prs{u_1, \ldots, u_k}$
בסיס של
$V_2$
ותהי
$C = \prs{w_1, \ldots, w_\ell}$
קבוצה סדורה של
$\ell \coloneqq \dim_{\mbb{F}}\prs{V_1}$
וקטורים.
אז
$B \uplus C$
בסיס של
$V$
אם ורק אם
$P(C) \coloneqq \prs{P{\prs{w_1}}, \ldots, P\prs{w_\ell}}$
בסיס של
$C$
כאשר
$P \in \End_{\mbb{F}}\prs{V}$
ההטלה על
$V_1$
במקביל ל־%
$V_2$.
כלומר, ההעתקה המקיימת
$P\prs{v_1 + v_2} = v_1$
לכל
$v_1 \in V_1, v_2 \in V_2$.
\end{lemma}

\begin{proof}
\begin{itemize}
\item נניח כי
$B \uplus C$
בסיס  של
$V$
ונראה כי
$P\prs{C}$
בסיס של
$V_1$.
יהיו
$\alpha_i \in \mbb{F}$
עבורם
\[\text{.} \sum_{i \in \brs{\ell}} \alpha_i P\prs{w_i} = 0\]
נראה כי
$\alpha_i = 0$
לכל
$i \in \brs{\ell}$.

מתקיים
\begin{align*}
\sum_{i \in \brs{\ell}} \alpha_i w_i
&=
\sum_{i \in \brs{\ell}} \alpha_i \prs{w_i - P\prs{w_i}} + \sum_{i \in \brs{\ell}} \alpha_i P\prs{w_i}
\\&=
\sum_{i \in \brs{\ell}} \alpha_i \prs{w_i - P\prs{w_i}}
\end{align*}
כאשר
$w_i - P\prs{w_i} \in V_2$
לכל
$i \in \brs{\ell}$.
קיבלנו צירוף לינארי של וקטורים ב־%
$C$
ששווה לוקטור ב־%
$\Span\prs{B}$,
אך
$B \uplus C$
הינו בסיס של
$V$
ולכן הביטוי הנ"ל שווה אפס.
לכן, מכך שהקבוצה
$C$
בלתי תלויה לינארית (כי $B \uplus C$ בסיס) נקבל כי
$\alpha_i = 0$
לכל
$i \in \brs{\ell}$,
כנדרש.

\item נניח כעת כי
$P\prs{C}$
הינו בסיס של
$V_1$.
בסיס של
$V$.

מתקיים כי
$P\prs{C}$
בסיס של
$V_1$
ו־%
$B$
בסיס של
$V_2$
כאשר
$V = V_1 \oplus V_2$,
לכן
$B \uplus P\prs{C}$
בסיס של
$V$.

יהיו
$\alpha_i, \beta_j \in \mbb{F}$
עבורם
\[\text{.} \sum_{i \in \brs{\ell}} \alpha_i w_i + \sum_{j \in \brs{k}} \beta_j u_j = 0\]
נראה כי
$\alpha_i = \beta_j = 0$
לכל
$i,j$.
אכן, מתקיים
\begin{align*}
0 &= \sum_{i \in \brs{\ell}} \alpha_i w_i + \sum_{j \in \brs{k}} \beta_j u_j
\\&=
\sum_{i \in \brs{\ell}} \alpha_i P\prs{w_i} + \sum_{i \in \brs{\ell}} \alpha_i \prs{w_i - P\prs{w_i}} + \sum_{j \in \brs{k}} \beta_j u_j
\end{align*}
אבל
$w_i - P\prs{w_i} \in V_2$
לכל
$i \in \brs{\ell}$
ולכן יש
$\gamma_j \in \mbb{F}$
עבורן
\[\sum_{i \in \brs{\ell}} \alpha_i \prs{w_i - P\prs{w_i}} = \sum_{j \in \brs{k}} \gamma_j u_j\]
ונקבל כי
\[0 = \sum_{i \in \brs{\ell}} \alpha_i P\prs{w_i} + \sum_{j \in \brs{k}} \prs{\beta_j + \gamma_j} u_j\]
ולכן כי
$\alpha_i = \beta_j + \gamma_j = 0$
לכל
$i,j$.
כיוון ש־%
$\alpha_i = 0$
לכל
$i$,
מתקיים כי
$\gamma_j = 0$
לכל
$j$
מאופן הגדרת
$\gamma_j$,
ולכן נקבל כי גם
$\beta_j = 0$,
כנדרש.
\end{itemize}
\end{proof}

\begin{proof}
נוכיח את הטענה באינדוקציה על
$n$.

\begin{description}
\item[בסיס האינדוקציה:]

עבור
$n = 1$
מתקיים כי $T$ אופרטור סקלרי נילפוטנטי, ולכן אופרטור האפס. אז
$k = 1$
ולכן
\[B = \prs{v_{1,1}}\]
בסיס של
$V$
וכן
$\Span\prs{B} = V$
כי
$v_{1,1}$
וקטור השונה מאפס.

\item[צעד האינדוקציה:]

נניח כעת שהטענה נכונה עבור כל מרחב־וקטורי ממימד קטן מ־%
$n$,
ונוכיח את הטענה עבור
$V$
שהינו ממימד
$n$.

לפי משפט מההרצאה, קיימים תת־מרחבים
$T$%
־שמורים
$V_1, \ldots, V_\ell \leq V$
עבורם
$V = \bigoplus_{i \in \brs{\ell}} V_i$
וכך ש־%
$T|_{V_i}$
הינו אי־פריד לכל
$i \in \brs{\ell}$.

מספר האיברים ב־$B_k$ הוא מספר הבלוקים מגודל
$k$
בצורת ז'ורדן של
$T$,
שזהו מספר הערכים
$i$
עבורם
$\dim_{\mbb{F}} V_i = k$,
כי עבור בסיסים
$D_i$
ל־%
$V_i$
ועבור
$D \coloneqq \biguplus_{i \in \brs{\ell}} D_i$
מתקיים
\begin{align*}
\text{.} \brs{T}_D = \pmat{\brs{T|_{V_1}}_{D_1} & 0 & \cdots & \cdots & 0 \\
0 & \brs{T|_{V_2}}_{D_2} & 0 & \ddots & \vdots \\
\vdots & \ddots & \ddots & \ddots & \vdots \\
\vdots & \ddots & \ddots & \brs{T|_{V_{\ell-1}}}_{D_{\ell-1}} & 0 \\
0 & \cdots & \cdots & 0 & \brs{T|_{V_\ell}}_{D_\ell}}
\end{align*}

לכל
$d \in \brs{k}$
נסמן
\[\text{.} N_d \coloneqq \set{i \in \brs{k}}{\dim_{\mbb{F}} V_i = d}\]

כיוון שלכל
$T|_{V_i}$
יש צורת ז'ורדן, וכיוון שאלו אופרטורים אי־פרידים, קיימים וקטורים
$u_1, \ldots, u_{n_k}$
כך ש־%
$C_{u_1}, \ldots, C_{u_{n_k}}$
בסיסי ז'ורדן של
המרחבים
$V_i$
כך ש־%
$i \in N_k$.
מתקיים
$u_j \in \ker\prs{T^k} \setminus \ker\prs{T^{k-1}}$
לכל
$j \in n_k$
וכן
$\prs{u_1, \ldots, u_{n_k}}$
בלתי־תלויה לינארית כיוון שהסכום
$V = \bigoplus_{i \in \brs{\ell}} V_i$
ישר.
לכן
$\prs{u_1, \ldots, u_{n_k}}$
בסיס של
.$\ker\prs{T^k} \setminus \ker\prs{T^{k-1}}$
אז
\[\Span\prs{B_k} = \Span\prs{u_1, \ldots, u_{n_k}}\]
ולכן
\[\text{.} \Span\prs{\biguplus_{j \in \brs{n_k}} C_{v_{k,j}}} = \Span\prs{\biguplus_{j \in\ \brs{n_k}} C_{u_j}} = \bigoplus_{i \in N_k} V_i\]

כעת, עבור כל
$i < k$
נגדיר
\begin{align*}
\tilde{B}_i \coloneqq P\prs{B_i}
\end{align*}
כאשר
$P$
ההטלה על
$\bigoplus_{i \notin \brs{\ell}} V_i$
במקביל ל־%
$\bigoplus_{i \in \brs{\ell}} V_i$,
וכאשר
$P\prs{v_1, \ldots, v_m} = \prs{P\prs{v_1}, \ldots, P\prs{v_k}}$
עבור כל קבוצה סדורה
$\prs{v_1, \ldots, v_k}$.
לפי למה
\ref{lemma:basis-projection}
הקבוצות הסדורות
$\tilde{B}_i = P\prs{B_i}$
הן בסיסים עבור
\[\ker\prs{T^i} \setminus \ker\prs{T^{i-1}}\]
בהתאמה.
אז לפי הנחת האינדוקציה
\[\biguplus_{i \in \brs{k-1}} \tilde{B}_i = P\prs{\biguplus_{i \in \brs{k-1}} B_i}\]
בסיס ז'ורדן עבור
$T|_{\biguplus_{i \notin N_k} V_i}$,
ולכן, לפי למה
\ref{lemma:basis-projection},
\[B = \biguplus_{i \in \brs{k}} B_i\]
הינו בסיס
$V$,
שמאופן בנייתו הינו בסיס ז'ורדן עבור
$T$.
\end{description}
\end{proof}

\end{document}