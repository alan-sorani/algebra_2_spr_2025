\documentclass[a4paper,10pt,twoside,openany]{article}

\usepackage[lang=hebrew]{maths}
\usepackage{hebrewdoc}
\usepackage{stylish}
\usepackage{lipsum}
\let\bs\blacksquare

\setlength{\parindent}{0pt}

%%%%%%%%%%%%
% Styling %
%%%%%%%%%%%%

\usepackage{enumitem}

%%%%%%%%%%%%%
% Counters  %
%%%%%%%%%%%%%

\setcounter{section}{0}     
            
%BIBLIOGRAPHY
\usepackage[
backend=biber,
style=alphabetic,
]{biblatex}
\addbibresource{bibliography.bib} %Imports bibliography file

%%%%%%%%%%
% Title  %
%%%%%%%%%%
\title{
אלגברה ב' - טענה בנוגע לדרך מציאת בסיס ז'ורדן
}
\date{}

\begin{document}
\maketitle

\begin{propositionstarred}
יהי
$V$
מרחב וקטורי ממימד סופי $n$ ומעל שדה
$\mbb{F}$,
יהי
$T \in \End_{\mbb{F}}\prs{V}$
אופרטור נילפוטנטי עם אינדקס נילפוטנטיות
$k$.
לכל
$i \in \brs{n}$
נסמן
$n_i = \dim \ker\prs{T^i} - \dim \ker\prs{T^{i-1}}$
שזהו מספר הבלוקים בצורת ז'ורדן של
$T$
מגודל לפחות
$i$.
יהי
$B_1 = \prs{v^{\prs{1}}_1, \ldots, v^{\prs{1}}_{n_k}}$
בסיס לגרעין
$\ker\prs{T}$
ונבנה באופן אינדוקטיבי קבוצות סדורות
$B_i = \prs{v^{\prs{i}}_1, \ldots, v^{\prs{i}}_{n_i}}$
כך שלכל
$i$
מתקיים כי
$\biguplus_{i \in \brs{i}} B_i$
בסיס עבור
$\ker\prs{T^i}$.
כלומר,
$B_i$
משלימה בסיס של
$\ker\prs{T^{i-1}}$
לבסיס של
$\ker\prs{T^i}$
לכל
$i \in \brs{k}$.

לכל
$v \in V$
עבורו
$T^\ell\prs{v} = 0$
אבל
$T^{\ell - 1}\prs{v} \neq 0$
נסמן את שרשרת ז'ורדן
$C_v = \prs{T^{\ell-1}\prs{v}, \ldots, T\prs{v}, v}$
וכן את המרחב הנפרש על ידה
$V_v = \Span\prs{C_v}$.

אז ניתן לקבל בסיס ז'ורדן עבור
$T$
באופן הבא.

\begin{enumerate}
\item
ניקח
$B = \biguplus_{v \in B_k} C_v$
וניקח
$i = k$.

\item
נקטין את
$i$
באחת.
נבחר תת־קבוצה סדורה מקסימלית
$\tilde{B}_i$
של
$B_i$
שהינה בלתי־תלויה לינארית בוקטורים ב־%
$B$
היא תהיה מגודל
$2\dim \ker\prs{T^i} - \dim\ker\prs{T^{i+1}} - \dim\ker\prs{T^{i-1}}$ כי זה מספר הבלוקים מגודל $i$ בצורת ז'ורדן של $T$.
לכל וקטור
$v \in \tilde{B}_i$
נשרשר ל־%
$B$
את
$C_v$.
כלומר נעדכן
\[\text{.} B_{\text{new}} = B_{\text{old}} \uplus C_v\]

\item
אם
$B$
באורך
$n$,
נסיים. אחרת נחזור לשלב הקודם.
\end{enumerate}

\end{propositionstarred}

לשם הוכחת הטענה, נזכיר תחילה הגדרה וטענה מההרצאה, ונוכיח למה על סכומים ישרים.

\begin{definition}
יהי
$V$
מרחב וקטורי סוף־מימדי מעל שדה $\mbb{F}$ ויהיו
$V_1, V_2 \leq V$
תת־מרחבים וקטוריים עבורם
$V = V_1 \oplus V_2$.
ההטלה על
$V_1$
במקביל ל־%
$V_2$
היא ההעתקה
$P_{V_1, V_2} \in \End_{\mbb{F}}\prs{V}$
היחידה המקיימת
$P_{V_1, V_2}\prs{v_1 + v_2} = v_1$
לכל
$v_1 \in V_1$
ולכל
$v_2 \in V_2$.
\end{definition}

\begin{proposition} \label{proposition:direct-sums}
יהי
$V$
מרחב וקטורי סוף־מימדי מעל שדה
$\mbb{F}$
ויהיו
$V_1, \ldots, V_m \leq V$
תת־מרחבים וקטוריים של $V$.
התנאים הבאים שקולים.

\begin{enumerate}
\item $V = \bigoplus_{i \in \brs{m}} V_i$;
\item לכל בחירה של בסיסים $B_i$ עבור כל $V_i$ בהתאמה, הקבוצה הסדורה
$\biguplus_{i \in \brs{m}} B_i$
הינה בסיס של
$V$;
\item
קיימים בסיסים
$B_i$
לכל
$V_i$
בהתאמה כך שהקבוצה הסדורה
$\biguplus_{i \in \brs{m}} B_i$
היא בסיס של
$V$.
\end{enumerate}
\end{proposition}

\begin{lemma} \label{lemma:basis-projection}
יהי
$V$
מרחב וקטורי סוף־מימדי מעל שדה
$\mbb{F}$
ויהיו
$V_1, V_2 \leq V$
תת־מרחבים עבורם
$V = V_1 \oplus V_2$.

יהי
$B = \prs{u_1, \ldots, u_k}$
בסיס של
$V_1$
ותהי
$C = \prs{w_1, \ldots, w_\ell}$
קבוצה סדורה של
$\ell \coloneqq \dim_{\mbb{F}}\prs{V_2}$
וקטורים.
אז
$B \uplus C$
בסיס של
$V$
אם ורק אם
\[P_{V_2, V_1}(C) \coloneqq \prs{P_{V_2, V_1}{\prs{w_1}}, \ldots, P_{V_2, V_1}\prs{w_\ell}}\]
בסיס של
$V_2$.
\end{lemma}

\begin{proof}
\begin{itemize}
\item נניח כי
$B \uplus C$
בסיס  של
$V$
ונראה כי
$P_{V_2, V_1}\prs{C}$
בסיס של
$V_2$.
יהיו
$\alpha_i \in \mbb{F}$
עבורם
\[\text{.} \sum_{i \in \brs{\ell}} \alpha_i P_{V_2, V_1}\prs{w_i} = 0\]
נראה כי
$\alpha_i = 0$
לכל
$i \in \brs{\ell}$.

מתקיים
\begin{align*}
\sum_{i \in \brs{\ell}} \alpha_i w_i
&=
\sum_{i \in \brs{\ell}} \alpha_i \prs{w_i - P_{V_2, V_1}\prs{w_i}} + \sum_{i \in \brs{\ell}} \alpha_i P_{V_2, V_1}\prs{w_i}
\\&=
\sum_{i \in \brs{\ell}} \alpha_i \prs{w_i - P_{V_2, V_1}\prs{w_i}}
\end{align*}
כאשר
\[w_i - P_{V_2, V_1}\prs{w_i} \in \im\prs{\id_V - P_{V_2, V_1}} \subseteq V_1\]
לכל
$i \in \brs{\ell}$.
קיבלנו צירוף לינארי של וקטורים ב־%
$C$
ששווה לוקטור ב־%
$\Span\prs{B}$,
אך
$B \uplus C$
הינו בסיס של
$V$
ולכן הביטוי הנ"ל שווה אפס.
לכן, מכך שהקבוצה הסדורה
$C$
בלתי תלויה לינארית (כי $B \uplus C$ בסיס) נקבל כי
$\alpha_i = 0$
לכל
$i \in \brs{\ell}$,
כנדרש.

\item נניח כעת כי
$P_{V_2, V_1}\prs{C}$
הינו בסיס של
$V_2$.

מטענה
\ref{proposition:direct-sums}
נקבל כי
$B \uplus P_{V_2, V_1}\prs{C}$
בסיס של
$V$.

יהיו
$\alpha_i, \beta_j \in \mbb{F}$
עבורם
\[\text{.} \sum_{i \in \brs{k}} \alpha_i u_i + \sum_{j \in \brs{\ell}} \beta_j w_j = 0\]
נראה כי
$\alpha_i = \beta_j = 0$
לכל
$i,j$.
אכן, מתקיים
\begin{align*}
0 &= \sum_{i \in \brs{k}} \alpha_i u_i + \sum_{j \in \brs{\ell}} \beta_j w_j
\\&=
\sum_{i \in \brs{k}} \alpha_i u_i + \sum_{j \in \brs{\ell}} \beta_j P_{V_2, V_1}\prs{w_j} + \sum_{j \in \brs{\ell}} \beta_j \prs{w_j - P_{V_2, V_1}\prs{w_j}}
\end{align*}
אבל
$w_j - P_{V_2, V_1}\prs{w_j} \in V_1$
לכל
$j \in \brs{\ell}$
ולכן גם
\[\text{.} \sum_{j \in \brs{\ell}} \beta_j \prs{w_j - P_{V_2, V_1}\prs{w_j}} \in V_1\]
אז יש
$\gamma_j \in \mbb{F}$
עבורן
\[\sum_{j \in \brs{\ell}} \beta_j \prs{w_j - P_{V_2, V_1}\prs{w_j}} = \sum_{i \in \brs{k}} \gamma_i u_i\]
ונקבל כי
\[0 = \sum_{i \in \brs{k}} \prs{\alpha_i + \gamma_i} u_i + \sum_{j \in \brs{\ell}} \beta_j P_{V_2, V_1}\prs{w_j}\]
אבל ראינו כי
$B \uplus P_{V_2, V_1}\prs{C}$
הינו בסיס עבור
$V$,
ולכן כל המקדמים שווים לאפס. כלומר,
$\alpha_i + \gamma_i = 0$
וכן
$\beta_j = 0$
לכל
$i,j$.
מכך ש־%
$\beta_j = 0$
לכל
$j$
נקבל מהגדרת
$\gamma_i$
כי
\[\sum_{i \in \brs{k}} \gamma_i u_i = 0\]
ולכן כיוון ש־%
$B = \prs{u_1, \ldots, u_k}$
בסיס נקבל כי
$\gamma_i = 0$
לכל
$i$.
אז מכך ש־%
$\alpha_i + \gamma_i = 0$
נקבל כי גם
$\alpha_i = 0$
לכל
$i \in \brs{k}$,
כנדרש.
\end{itemize}
\end{proof}

\begin{proof}[הוכחת הטענה]
נוכיח את הטענה באינדוקציה על
$n$.

\begin{description}
\item[בסיס האינדוקציה:]

עבור
$n = 1$
מתקיים כי $T$ אופרטור סקלרי נילפוטנטי, ולכן אופרטור האפס. אז לכל בחירה $\prs{v}$ עבור בסיס של
$\ker\prs{T}$
נקבל כי
$B = \prs{v}$
בסיס ז'ורדן.

\item[צעד האינדוקציה:]

נניח כעת שהטענה נכונה עבור כל מרחב־וקטורי ממימד קטן מ־%
$n$,
ונוכיח את הטענה עבור
$V$
שהינו ממימד
$n$.

לפי משפט מההרצאה, קיימים תת־מרחבים
$T$%
־שמורים
$V_1, \ldots, V_\ell \leq V$
עבורם
$V = \bigoplus_{i \in \brs{\ell}} V_i$
וכך ש־%
$T|_{V_i}$
הינו אי־פריד לכל
$i \in \brs{\ell}$.
נניח בלי הגבלת הכלליות כי המרחבים
$V_i$
עבורם
$\dim_{\mbb{F}}\prs{V_i} = k$
הם
$V_1, \ldots, V_{n_k}$
(נזכיר כי
$n_k$
הוא מספר הבלוקים מגודל
$k$
בצורת ז'ורדן של
$T$).
נסמן
$\tilde{W} = \bigoplus_{i \in \brs{n_k}} V_i$
$\hat{W} = \bigoplus_{i \in \brs{\ell} \setminus \brs{n_k}} V_i$.

מתקיים כי לכל
$i \in \brs{n_k}$
הצמצום
$T|_{V_i}$
הוא אופרטור נילפוטנטי אי־פריד על מרחב וקטורי ממימד
$k$,
ולכן יש לו צורת ז'ורדן
$J_k\prs{0}$.
לכן, קיימים וקטורים
$w_1, \ldots, w_{n_k}$
עבורם
$C_{w_i}$
בסיס ז'ורדן של
$V_i$,
לכל
$i \in \brs{n_k}$.
כיוון שהסכום
$V = \bigoplus_{i \in \ell} V_i$
ישר, נקבל מטענה
\ref{proposition:direct-sums}
כי
\[B_W \coloneqq \biguplus_{i \in \brs{n_k}} C_{w_i} = \prs{T^{k-1}\prs{w_1}, \ldots, T\prs{w_1}, w_1, \ldots, T^{k-1}\prs{w_{n_k}}, \ldots, T\prs{w_{n_k}}, w_{n_k}}\]
בסיס של
$W$.
\end{description}
\end{proof}

\end{document}