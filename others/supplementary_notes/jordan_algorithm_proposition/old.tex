\documentclass[a4paper,10pt,twoside,openany]{article}

\usepackage[lang=hebrew]{maths}
\usepackage{hebrewdoc}
\usepackage{stylish}
\usepackage{lipsum}
\let\bs\blacksquare

\setlength{\parindent}{0pt}

%%%%%%%%%%%%
% Styling %
%%%%%%%%%%%%

\usepackage{enumitem}

%%%%%%%%%%%%%
% Counters  %
%%%%%%%%%%%%%

\setcounter{section}{0}     
            
%BIBLIOGRAPHY
\usepackage[
backend=biber,
style=alphabetic,
]{biblatex}
\addbibresource{bibliography.bib} %Imports bibliography file

%%%%%%%%%%
% Title  %
%%%%%%%%%%
\title{
אלגברה ב' - טענה בנוגע לדרך מציאת בסיס ז'ורדן
}
\date{}

\begin{document}
\maketitle

\begin{propositionstarred}
יהי
$T \in \End_{\mbb{F}}\prs{V}$
אופרטור נילפוטנטי על מרחב סוף־מימדי
$V$,
כך שמתקיים
$r_a\prs{0} = \dim_{\mbb{F}}\prs{V}$.
יהיו
$v_1, \ldots, v_k \in V$
וגם
$\ell_1, \ldots, \ell_k \in \mbb{N}_+$
עבורם
$v_i \in \ker\prs{T^{\ell_i}} \setminus \ker\pr{T^{\ell_i - 1}}$.

נסמן
\begin{align*}
B \coloneqq \prs{T^{\ell_1 - 1}\prs{v_1}, T^{\ell_1 - 2}\prs{v_1}, \ldots, T\prs{v_1}, v_1, \ldots, T^{\ell_k - 1}\prs{v_k}, T^{\ell_k - 2}\prs{v_k}, \ldots, T\prs{v_k}, v_k}
\end{align*}
ויהיו
$v \notin \Span\prs{B}$
ו־%
$\ell \leq \min\set{\ell_1, \ldots, \ell_k}$
שלם
עבורם
$T^\ell\prs{v} = 0$
וגם
$T^{\ell-1}\prs{v} \neq 0$.
נסמן
$C = \prs{T^{k-1}\prs{v}, T^{k-2}\prs{v}, \ldots, T\prs{v}, v}$.
אז
$B \uplus C$
קבוצה סדורה בלתי־תלויה לינארית.
\end{propositionstarred}

כדי להוכיח את הטענה, ראשית נזכיר%
\textbackslash
נוכיח למה עבור משלימים ישרים.

\begin{lemma} \label{lemma:complete-from-basis}
יהי
$V$
מרחב וקטורי סוף־מימדי מעל שדה
$\mbb{F}$
ויהי
$U \leq V$
תת־מרחב עם בסיס
$B$.
יהי
$C$
בסיס של
$V$.
אז:

\begin{enumerate}
\item ניתן להשלים את
$B$
לבסיס של
$V$
על ידי הוספת וקטורים מ־%
$C$.

\item קיים משלים ישר
$W$
של
$U$
עם בסיס של וקטורים מ־%
$C$.
\end{enumerate}
\end{lemma}

\begin{proof}
\begin{enumerate}
\item נסמן
$n \coloneqq \dim_{\mbb{F}}\prs{V}$
ונוכיח את הטענה באינדוקציה על
$m = n - \abs{B}$.

עבור
$m = 0$
מתקיים
$\abs{B} = n$
ולכן
$U = V$.
נניח שהטענה נכונה לכל
$k < m$
ונוכיח אותה עבור
$m$.

אם
$C \subseteq U$,
מתקיים
\[V = \Span_{\mbb{F}}\prs{C} \subseteq \Span_{\mbb{F}}\prs{U} = U\]
ולכן
$V = U$
בסתירה לכך שהמימדים שונים.
לכן, קיים
$c \in C \setminus U$.
אז
$B \cup \prs{c}$
קבוצה בלתי־תלויה לינארית, כי
$c$
אינו צירוף לינארי של הוקטורים הקודמים. נגדיר
$U' = \Span_{\mbb{F}}\prs{B \cup \prs{c}}$.
אז
\[n - \dim\prs{U'} = n - \abs{B} - 1 = m-1 < m\]
ולכן ניתן להשתמש בהנחת האינדוקציה ולקבל שניתן להשלים את
$B \cup \prs{c}$
לבסיס
$\prs{B \cup \prs{c}} \cup \prs{c_2, \ldots, c_m}$
של
$V$,
כאשר
$c_i \in C$.
אז
$c, c_2, \ldots, c_m \in C$
משלימים את
$B$
לבסיס של
$V$.

\item בסימונים של הסעיף הקודם,
$B \cup \prs{c, \ldots, c_m}$
בסיס של
$V$.
נסמן
$D = \prs{c, c_2, \ldots, c_m}$
וגם
$W = \Span_{\mbb{F}}\prs{D}$.
אז
$B \cup D$
בסיס של
$V$
ולכן
\[\text{,} V = \Span_{\mbb{F}}\prs{B} \oplus \Span_{\mbb{F}}\prs{D} = U \oplus W\]
כנדרש.
\end{enumerate}
\end{proof}

מכך נסיק את הלמה הבאה.

\begin{lemma} \label{lemma:direct-subsum}
יהי
$V$
מרחב וקטורי סוף־מימדי מעל שדה
$\mbb{F}$,
ויהיו
$V_1, V_2 \leq V$
תת־מרחבים עבורם
$V = V_1 + V_2$.
קיים
$\tilde{V}_2 \leq V$
עבורו
$V = V_1 \oplus \tilde{V}_2$.
\end{lemma}

\begin{proof}
יהי
$B$
בסיס ל־%
$V_1$
ויהי
$\hat{C}$
בסיס ל־%
$V_2$.
נשלים את
$\hat{C}$
לבסיס
$C$
של
$V$.

לפי למה
\ref{lemma:complete-from-basis}
ניתן להשלים את
$B$
לבסיס
$B \uplus \tilde{C}$
של
$V$
כאשר וקטורי
$\tilde{C}$
נמצאים ב־%
$C$.
לכן
\[\tilde{V}_2 \coloneqq \Span\prs{\tilde{C}} \leq \Span\prs{C} = V_2\]
וכן
$V = \Span\prs{B} \oplus \tilde{C}$
כיוון שמטענה מההרצאה מתקיים עבור בסיסים
$B_1, B_2$
של תת־מרחבים
$W_1, W_2$
בהתאמה כי
$B_1 \uplus B_2$
בסיס של
$V$
אם ורק אם
$V = W_1 \oplus W_2$.
\end{proof}

\begin{lemma} \label{lemma:oplus-dimker}
יהי
$V$
מרחב וקטורי סוף־מימדי מעל שדה סופי
$\mbb{F}$,
יהי
$T \in \End_{\mbb{F}}\prs{V}$,
ויהיו
$V_1, V_2 \leq V_2$
תת־מרחבים
$T$%
־שמורים
של
$V$
עבורם
$V = V_1 + V_2$
וגם
$V_2 \leq \ker\prs{T}$.
אז
\[\text{.} \dim \ker\prs{T} = \dim\ker\prs{T|_{V_1}} + \dim\prs{V_2} - \dim \prs{V_1 \cap V_2}\]
\end{lemma}

\begin{proof}
מלמה
\ref{lemma:direct-subsum}
קיים
$\tilde{V}_2 \leq V_2$
עבורו
$V = V_1 \oplus \tilde{V}_2$.
המרחב
$\tilde{V}_2$
הינו
$T$%
־שמור כיוון שהוא מוכל ב־%
$\ker\prs{T}$.
ראינו (בתרגיל בית) כי כאשר
$V$
סכום ישר של מרחבים
$T$%
־שמורים, הגרעין
$\ker\prs{T}$
הוא הסכום הישר של הצמצומים של
$T$
לאותם מרחבים, לכן
\[\text{.} \ker\prs{T} = \ker\prs{T|_{V_1}} \oplus \dim\ker\prs{T|_{\tilde{V}_2}}\]
אבל
$\tilde{V}_2 \leq \ker\prs{T}$
ולכן
\[\dim\ker\prs{T|_{\tilde{V}_2}} = \dim\prs{\tilde{V}_2} = \dim\prs{V_2} - \dim\prs{V_1 \cap V_2}\]
כאשר השוויון האחרון מתקיים מאי שוויון המימדים.
מכך שהמימד של סכום ישר הוא סכום המימדים, נקבל את הנדרש.
\end{proof}

כיוון שעבור שרשרת ז'ורדן
$C$
אמור להתאים בלוק, נצפה שהוקטור העצמי
$T^{k-1}\prs{v}$
אינו תלוי לינארית בוקטורי
$B$,
כי הוספת בלוק מצריכה הגדלה של הריבוי הגיאומטרי. זה מוביל אותנו ללמה הבאה.

\begin{lemma}\label{lemma:extra-eigenvector}
תחת תנאי הטענה וההנחה שבצורת ז'ורדן של
$T|_{W}, T|_{W_B}$
גדלי הבלוקים מגודל לפחות
$k+1$
זהים,
מתקיים כי
$B \uplus T^{k-1}\prs{v}$
בלתי־תלויה לינארית.
\end{lemma}

\begin{proof}
ראשית,
\[\text{,} \dim\prs{W} \geq \dim \Span\prs{B \uplus \prs{v}} = \dim\prs{B} + 1\]
וכיוון שגדלי הבלוקים מגודל לפחות
$k+1$
זהים בשתי צורות הז'ורדן נקבל כי בצורת ז'ורדן של
$T|_W$
יש לפחות בלוק אחד יותר מאשר בצורת ז'ורדן של
$T|_{W_B}$.
כלומר כי,
$\dim\ker\prs{T|_W} > \dim\ker\prs{W_B}$.

לכן קיים וקטור
$w \in W \setminus W_B$
עבורו
$T\prs{w} = 0$.
נכתוב
\[w = \sum_{i=0}^{k-1} \alpha_i T^i\prs{v} + \sum_{u \in B} \beta_u u\]
עבור
$\alpha_i, \beta_u \in \mbb{F}$
וכאשר
$\alpha_i$
לא כולם אפס.

אם
$\alpha_i = 0$
לכל
$i < k-1$
נקבל כי
\[\sum_{u \in B} \beta_u T\prs{u} = 0\]
ולכן
$\beta_0 = 0$
לכל
$u \in B \setminus \ker\prs{T}$.
במקרה זה
\[w = T^{k-1}\prs{v} + \sum_{u \in B \cap \ker\prs{T}} \beta_u u\]
אינו נמצא ב־%
$W_B$
ולכן
$B \uplus T^{k-1}\prs{v}$
בלתי־תלויה לינארית כנדרש.

נניח אם כן שקיים
$0 \leq j < k-1$
עבורו
$\alpha_j \neq 0$,
ונבחר את
$j$
המינימלי המקיים זאת.

נפעיל
$T^{k-j-1}$
ונקבל כי
\[\text{.} \alpha_j T^{k-1}\prs{v} = \sum_{i=0}^{k-1} \alpha_i T^{i+k-j-1}\prs{v} = -\sum_{u \in B} \beta_u T^{k-j-1}\prs{u}\]
\end{proof}

\begin{proof}[הוכחת הטענה]
יהיו
\begin{align*}
W_B &\coloneqq \Span\prs{B} \text{,} \\
W_C &\coloneqq \Span\prs{C} \text{,} \\
\text{.} \hphantom{a} W &\coloneqq W_B \oplus W_C = \Span\prs{B \uplus C}
\end{align*}

נוכיח את הטענה בשלבים הבאים:

\begin{enumerate}
\item
נראה שגדלי הבלוקים מגודל לפחות
$k+1$
זהים בין צורות ז'ורדן של
$T|_{W_B}$
ושל
$T|_W$.

\item
נראה שמספר הבלוקים מגודל
$k$
בצורת ז'ורדן של
$T|_W$
הוא אחד יותר מזה שבצורת ז'ורדן של
$T|_{W_B}$.

\item
נסיק כי בצורת ז'ורדן של
$T|_W$
אותם גדלי בלוקים כמו בצורת ז'ורדן של
$T|_{W_B}$,
פרט לכך שיש בלוק נוסף מגודל
$k$,
ונסיק מכך כי
$B \uplus C$
בסיס של
$W$
ולכן קבוצה סדורה בלתי־תלויה לינארית.
\end{enumerate}

נוכיח כעת את הטענה לפי השלבים.

\begin{enumerate}
\item %1

הוקטורים ב־C כולם נמצאים ב־%
$\ker\prs{T^k}$
ולכן גם ב־%
$\ker\prs{T^{k+1}}$.
לכן נצפה שיתקיים
\[\text{,}\dim \ker\prs{T|_W^{k+1}} - \dim \ker\prs{T|_W^k} = \dim \ker\prs{T|_{W_B}^{k+1}} - \dim \ker\prs{T|_{W_B}^k}\]
כלומר שמספר בלוקי ז'ורדן מגודל לפחות
$k+1$
זהה בין צורות ז'ורדן של
$T|_{W_B}$
ושל
$T|_W$.
זה אכן מתקיים מלמה
\ref{lemma:oplus-dimker}
עבור
$T|_W^k, T|_W^{k+1}$,
כיוון ששני הנסכמים באגף ימין גדולים מאלו שבאגף שמאל באותו קבוע.

\item %2

נסמן
\[\hat{C} = \prs{T^{k-1}\prs{v}, T^{k-2}\prs{v}, \ldots, T\prs{v}}\]
וגם
\[\text{.} W_{\hat{C}} \coloneqq \Span\prs{\hat{C}}\]
מלמה
\ref{lemma:oplus-dimker}
נקבל כי
\begin{align*}
\dim \ker \prs{T|_W^{k-1}} &= \dim \ker \prs{T|_{W_B}^{k-1}} + k - 1 - \dim\prs{W_B \cap W_{\hat{C}}} \\
\dim \ker \prs{T|_W^k} &= \dim \ker \prs{T|_{W_B}^k} + k - \dim\prs{W_B \cap W_C}
\end{align*}
ומהחסרת המשוואות נקבל כי
\begin{align*}
\dim \ker \prs{T|_W^k} - \dim \ker \prs{T|_W^{k-1}} &=
\dim \ker \prs{T|_{W_B}^k} - \dim \ker \prs{T|_{W_B}^{k-1}} 
\\& +
\dim\prs{W_B \cap W_{\hat{C}}} - \dim\prs{W_B \cap W_C}
\\ \text{.} \hphantom{\dim \ker \prs{T|_W^k} - \dim \ker \prs{T|_W^{k-1}}} & +
1
\end{align*}
לכן נותר להראות כי
$\dim\prs{W_B \cap W_{\hat{C}}} = \dim\prs{W_B \cap W_C}$,
כלומר כי
$W_B \cap W_{\hat{C}} = W_B \cap W_C$.

נניח בדרך השלילה שקיים
$u \in W_B \cap W_C \setminus W_B \cap W_{\hat{C}}$.
אז ניתן לכתוב
\begin{align*}
u &= \sum_{i=0}^{k-1} = \alpha_i T^i\prs{v} \\
u &= \sum_{w \in B} \beta_w w
\end{align*}
עבור סקלרים
$\alpha_i, \beta_w \in \mbb{F}$
כאשר
$\alpha_0 \neq 0$.
נקבל כי
\[\text{.} \sum_{i=0}^{k-1} = \alpha_i T^i\prs{v} = \sum_{w \in B} \beta_w w\]

נפעיל את
$T^{k-1}$
על שני האגפים ונקבל כי
\[ \alpha_0 T^{k-1}\prs{v} = \sum_{w \in B} \beta_w T^{k-1}\prs{w} \]
כאשר הוקטורים
$T^{k-1}\prs{w}$
כולם שייכים ל־%
$B$
מאופן הגדרתו.
נחלק ב־%
$\alpha_0$
ונקבל כי
\[ \text{.} T^{k-1}\prs{v} = \sum_{w \in B} \frac{\beta_w}{\alpha_0} T^{k-1}\prs{w} \]

\item %3

הראנו כי גדלי הבלוקים בצורת ז'ורדן של
$T|_W$
הם בדיוק אלו בצורת ז'ורדן של
$T|_{W_B}$
פרט לכך שיש בלוק נוסף מגודל
$k$.
לכן
\[\text{.} \dim\prs{W} = \dim\prs{W_B} + k = \dim\prs{W_B} + \dim\prs{W_C}\]
ממשפט המימדים מתקיים כי
\[\dim\prs{W} = \dim\prs{W_B} + \dim\prs{W_C} - \dim\prs{W_B \cap W_C}\]
ולכן נקבל כי
$\dim\prs{W_B \cap W_C} = 0$
ולכן הסכום
$W = W_B + W_C$
הינו סכום ישר.
קיבלנו כי
$W = \Span\prs{B} \oplus \Span\prs{C}$
מה ששקול לכך ש־%
$B \uplus C$
בסיס של
$W$,
ובפרט קבוצה סדורה בלתי־תלויה לינארית.
\end{enumerate}

\end{proof}

\end{document}