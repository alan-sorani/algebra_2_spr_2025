\documentclass[a4paper,10pt,twoside,openany]{article}

\usepackage[lang=hebrew]{maths}
\usepackage{hebrewdoc}
\usepackage{stylish}
\usepackage{lipsum}
\let\bs\blacksquare

\setlength{\parindent}{0pt}

%%%%%%%%%%%%
% Styling %
%%%%%%%%%%%%

\usepackage{enumitem}

%%%%%%%%%%%%%
% Counters  %
%%%%%%%%%%%%%

\setcounter{section}{1}     
            
%BIBLIOGRAPHY
\usepackage[
backend=biber,
style=alphabetic,
]{biblatex}
\addbibresource{bibliography.bib} %Imports bibliography file

%%%%%%%%%%
% Title  %
%%%%%%%%%%
\title{
אלגברה ב' - הצעה לפתרון מועד ב'
}
\date{אביב 2025}

\begin{document}
\maketitle

\setcounter{question}{4}

\begin{question}
נסמן
$\mbb{R}\brs{x}_{\leq 2} \coloneqq \set{p\prs{x} \in \mbb{R}\brs{x}}{\deg\prs{p\prs{x}} \leq 2}$,
ונגדיר מכפלה פנימית על
$\mbb{R}\brs{x}_{\leq 2}$
על ידי
$\trs{p\prs{x}, q\prs{x}} \coloneqq \int_0^1 p\prs{t} q\prs{t} \diff t$
לכל
$p\prs{x}, q\prs{x} \in \mbb{R}\brs{x}_{\leq 2}$.
מצאו בסיס אורתונורמלי
$\prs{p_0\prs{x}, p_1\prs{x}, p_2\prs{x}}$
של
$\mbb{R}\brs{x}_{\leq 2}$
כך ש־%
$\Span\prs{p_0\prs{x}, p_1\prs{x}} = \Span\prs{1,x}$.
\end{question}

\begin{solution}
נסתכל על הבסיס
$E \coloneqq \prs{u_0, u_1, u_2} = \prs{1, x, x^2}$
של
$\mbb{R}\brs{x}_{\leq 2}$.
תהליך גרם־שמידט על בסיס
$\prs{v_1, \ldots, v_n}$
של מרחב מכפלה פנימית נותן בסיס אורתונורמלי
$\prs{e_1, \ldots, e_n}$
של אותו מרחב כך שמתקיים
\[\text{.} \forall i \in \brs{n} : \Span\prs{e_1, \ldots, e_i} = \Span\prs{v_1, \ldots, v_i}\]
לכן, תהליך גרם־שמידט על
$E$
יתן לנו בסיס אורתונורמלי
$\prs{p_0\prs{x}, p_1\prs{x}, p_2\prs{x}}$
של
$\mbb{R}\brs{x}_{\leq 2}$
עבורו
\[\text{,} \forall i \in \set{0, 1, 2} : \Span\prs{p_0\prs{x}, \ldots, p_i\prs{x}} = \Span\prs{v_0, \ldots, v_i}\]
ובפרט
\[\text{,} \Span\prs{p_0\prs{x}, p_1\prs{x}} = \Span\prs{v_0, v_1} = \Span\prs{1,x}\]
כנדרש.

נבצע את תהליך גרם־שמידט על הבסיס
$E$.
בשלב הראשון, ננרמל את הוקטור הראשון בבסיס.
מתקיים
\[\norm{u_0}^2 = \norm{1}^2 = \int_0^1 1^2 \diff t = \left. x \right|_{x=0}^1 = 1\]
ולכן גם
$\norm{u_0} = 1$
ונקבל
\[\text{.} p_0\prs{x} = \frac{u_0}{\norm{u_0}} = \frac{1}{1} = 1\]
כעת, נחסר מ־%
$u_1$
את ההטלה שלו על המרחב הנפרש על ידי
$p_0\prs{x}$,
וננרמל את הוקטור שנקבל.

\begin{align*}
w_1 &\coloneqq u_1 - \trs{u_1, p_0\prs{x}} p_0\prs{x}
\\&=
x - \trs{x, 1} 1
\\&=
x - \int_0^1 x \diff x
\\&=
x - \left.\prs{\frac{x^2}{2}} \right|_{x=0}^1
\\&=
x - \frac{1}{2}
\end{align*}

ומתקיים

\begin{align*}
\norm{w_1}^2 &=
\int_0^1 w_1^2 \diff x
\\&=
\int_0^1 x^2 - x + \frac{1}{4} \diff x
\\&=
\left. \frac{x^3}{3} - \frac{x^2}{2} + \frac{1}{4} \right|_{x = 0}^1
\\&=
\frac{1}{3} - \frac{1}{2} + \frac{1}{4}
\\&=
\frac{1}{3} - \frac{1}{4}
\\&=
\frac{4}{12} - \frac{3}{12}
\\&=
\frac{1}{12}
\end{align*}
ואז
\begin{align*}
\text{.} \norm{w_1} = \frac{1}{\sqrt{12}}
\end{align*}
ננרמל את הוקטור
$w_1$
ונקבל
\begin{align*}
\text{.} p_1\prs{x} = \frac{w_1}{\norm{w_1}} = \sqrt{12} w_1 = \sqrt{12} \prs{x - \frac{1}{2}}
\end{align*}

כעת, חסר מ־%
$u_2$
את ההטלה שלו על המרחב הנפרש על ידי
$p_0\prs{x}, p_1\prs{x}$,
וננרמל את התוצאה כדי לקבל את
$p_2\prs{x}$.

\begin{align*}
w_2 &\coloneqq u_2 - \trs{u_2, p_0\prs{x}} p_0\prs{x} - \trs{u_2, p_1\prs{x}} p_1\prs{x}
\\&=
x^2 - \trs{x^2, 1} 1 - \trs{x^2, \sqrt{12} \prs{x - \frac{1}{2}}} \cdot \sqrt{12} \prs{x - \frac{1}{2}}
\\&=
x^2 - \trs{x^2, 1} - 12 \trs{x^2, x - \frac{1}{2}} \prs{x - \frac{1}{2}}
\end{align*}

נחשב את המכפלות הפנימיות המופיעות בביטוי שקיבלנו.

\begin{align*}
\text{,} \trs{x^2, 1} &= \int_0^1 x^2 \diff x = \left. \frac{x^3}{3} \right|_{x=0}^1 = \frac{1}{3}
\end{align*}
וכן
\begin{align*}
\trs{x^2, x - \frac{1}{2}} &= \int_0^1 x^3 - \frac{1}{2} x^2 \diff x
\\&=
\left. \frac{x^4}{4} - \frac{x^3}{6} \right|_{x=0}^1
\\&=
\frac{1}{4} - \frac{1}{6}
\\&=
\frac{3}{12} - \frac{2}{12}
\\\text{.} \hphantom{\trs{x^2, x - \frac{1}{2}}} &=
\frac{1}{12}
\end{align*}

נציב זאת בביטוי שקיבלנו עבור
$w_2$
ונקבל כי

\begin{align*}
w_2 &= x^2 - \frac{1}{3} - 12 \cdot \frac{1}{12} \prs{x - \frac{1}{2}}
\\&=
x^2 - x + \frac{1}{2} - \frac{1}{3}
\\ \text{.} \hphantom{w_2} &=
x^2 -x + \frac{1}{6}
\end{align*}

אז

\begin{align*}
\norm{w_2}^2 &=
\int_0^1 \prs{x^2 -x + \frac{1}{6}}_2^2 \diff x
\\&=
\int_0^1 x^4 - 2 x^3 + \frac{1}{3} x^2 + x^2 - \frac{1}{3} x + \frac{1}{36} \diff x
\\&=
\left. \frac{x^5}{5} - 2 \cdot \frac{x^4}{4} + \frac{x^3}{9} + \frac{x^3}{3} - \frac{x^2}{6} + \frac{x}{36} \right|_{x=0}^1
\\&=
\frac{1}{5} - \frac{1}{2} + \frac{1}{9} + \frac{1}{3} - \frac{1}{6} + \frac{1}{36}
\\&=
\frac{1}{5} - \frac{18}{36} + \frac{4}{36} + \frac{12}{36} - \frac{6}{36} + \frac{1}{36}
\\&=
\frac{1}{5} - \frac{7}{36}
\\&=
\frac{36 - 35}{180}
\\&=
\frac{1}{180}
\end{align*}
ולכן
\begin{align*}
\norm{w_2} = \frac{1}{\sqrt{180}}
\end{align*}
ואז
\begin{align*}
\text{.} p_2\prs{x} = \frac{w_2}{\norm{w_2}} = \sqrt{180} \prs{x^2 - x + \frac{1}{6}}
\end{align*}

\end{solution}

\setcounter{question}{7}

\newpage

\begin{question}
יהי
$V$
מרחב וקטורי סוף־מימדי מעל
$\mbb{C}$.
נסמן
$n \coloneqq \dim_{\mbb{C}}\prs{V}$
ויהי
$T \in \End_{\mbb{C}}\prs{V}$.
האם בהכרח מתקיים ש־%
$V = \ker\prs{T^n} \oplus \im\prs{T^n}$?
\end{question}

\begin{solution}
כן. נציג שתי דרכים לכך.

\begin{description}
\item[דרך 1:]

נראה כי
$\ker\prs{T^n} \cap \im\prs{T^n} = \set{0}$
ונקבל כי הסכום
$\ker\prs{T^n} \oplus \im\prs{T^n}$
ישר. אך המימד של סכום ישר הוא סכום המימדים, ולפי משפט המימדים
\[\dim \ker\prs{S} + \dim \im \prs{S} = \dim V\]
לכל אופרטור
$S \in \End_{\mbb{C}}\prs{V}$
ובפרט
\[\text{.} \dim\prs{\ker\prs{T^n} \oplus \im\prs{T^n}} = \dim \ker\prs{T^n} + \dim \im \prs{T^n} = \dim V\]
אז
$\ker\prs{T^n} \oplus \im\prs{T^n}$
תת־מרחב של
$V$
ממימד
$\dim V$
ולכן שווה ל־%
$V$,
כנדרש.

נניח בדרך השלילה שקיים
$y \in \ker\prs{T^n} \cap \im\prs{T^n} \setminus \set{0}$.
אז
$T^n\prs{y} = 0$
וגם קיים
$x \in V$
עבורו
$T^n\prs{x} = y$.
נקבל כי הסדרה
\[x, T(x), \ldots, T^{n-1}\prs{x}, y = T^n\prs{x}, T^{n+1}\prs{x}, \ldots, T^{2n}\prs{x} = T^n\prs{y}\]
מתאפסת החל ממקום מסוים כי $T^n\prs{y} = 0$, ונסמן ב־%
$\ell$
את האינדקס המינימלי עבורו
$T^m\prs{x} = 0$.
אז
$m > n$
כי
$T^n\prs{x} = y \neq 0$.
אז
\begin{align*}
T^{m-1}\prs{x} \neq 0 \\
T^m\prs{x} = 0
\end{align*}
ומלמה מההרצאה זה מראה כי הקבוצה הסדורה
\begin{align*}
\mcal{C} \coloneqq \prs{T^{m-1}\prs{x}, \ldots, T\prs{x}, x}
\end{align*}
הינה בלתי־תלויה לינארית.
אך זאת קבוצה סדורה מגודל
$m > n$
ואילו כל בסיס של
$V$
הינו מגודל
$n$,
בסתירה.

\item[דרך 2:]

לפי משפט ז'ורדן, מעל שדה סגור אלגברית, לכל אופרטור קיימת צורת ז'ורדן.
לכן, קיימים בסיס
$B = \prs{v_1, \ldots, v_n}$
של
$V$,
סקלרים
$\lambda_1, \ldots, \lambda_k \in \mbb{C}$
ושלמים חיוביים
$m_1, \ldots, m_k \in \mbb{N}_+$
עבורם
\[\text{.} \brs{T}_B = \diag\prs{J_{m_1}\prs{\lambda_1}, \ldots, J_{m_k}\prs{\lambda_k}}\]

אז
\[\text{.} \brs{T^n}_B = \brs{T}_B^n = \diag\prs{J_{m_1}\prs{\lambda_1}^n, \ldots, J_{m_k}\prs{\lambda_k}^n}\]
מטריצת ז'ורדן עם ערך עצמי שונה מאפס הינה הפיכה, לכן
$J_{m_i}\prs{\lambda_i}^n$
גם הן הפיכות לכל
$i \in \brs{k}$.
מטריצת ז'ורדן
$J_m\prs{0}$
הינה נילפוטנטית מסדר
$m$
וכיוון ש־%
$m_i \leq n$
לכל
$i \in \brs{k}$
נקבל כי
$J_{m_i}\prs{\lambda_i}^n = 0$
לכל
$i \in \brs{k}$
עבורו
$\lambda_i = 0$.

נבחר את
$B$
כך שבצורת ז'ורדן
$\brs{T}_B$
הבלוקים עם ערך עצמי
$0$,
אם קיימים, יופיעו בסוף, ונקבל כי
\[\brs{T^n}_B = \diag\prs{J_{m_1}\prs{\lambda_1}, \ldots, J_{m_{k'}}\prs{\lambda_{k'}}, 0_{\ell \times \ell}}\]
כאשר
$k'$
מספר הבלוקים עם ערך עצמי שונה מאפס בצורת ז'ורדן של
$T$
וכאשר
$\ell$
הריבוי האלגברי של
$0$
כערך עצמי של
$T$.

המטריצה
\[A \coloneqq \diag\prs{J_{m_1}\prs{\lambda_1}, \ldots, J_{m_{k'}}\prs{\lambda_{k'}}}\]
הפיכה כיוון שזאת מטריצה אלכסונית בלוקים עם בלוקים הפיכים, ומתקיים
\[\text{.} \brs{T^n}_B = \diag\prs{A, 0_{\ell \times \ell}}\]
כיוון שמימד הגרעין של מטריצה אלכסונית בלוקים הוא סכום מימדי הגרעין של הבלוקים השונים, נקבל כי
\[\text{.} \dim \ker\prs{ \brs{T^n}_B } = \dim \ker\prs{A} + \dim \ker\prs{0_{\ell \times \ell}} = \ell\]
כיוון ש־%
\[\brs{T^n}_B e_i = 0\]
לכל
$i \geq n - \ell + 1$,
שכן זאת העמודה ה־%
$i$
של
$\brs{T^n}_B$
וכי ה־%
$\ell$
העמודות האחרונות הן עמודות אפסים, נקבל כי
\begin{align*}
\Span \prs{e_{n - \ell + 1}, \ldots, e_n} \subseteq \ker\prs{\brs{T^n}_B}
\end{align*}
אך כיוון ששני אלו מרחבים וקטוריים ממימד
$\ell$
נקבל שוויון.

כעת,
\[\brs{T^n}_B \brs{v}_B = \brs{T^n\prs{v}}_B\]
ולכן
$\ker\prs{T^n}$
נפרש על ידי וקטורים ש־%
$\prs{e_{n - \ell + 1}, \ldots, e_n}$
הם וקטורי הקואורדינטות שלהם בבסיס
$B$.
נקבל כי
$\prs{v_{n - \ell + 1}, \ldots, v_n}$
בסיס של
$\ker\prs{T^n}$.

נשים לב כי אף אחד מהוקטורים ב־%
$\Span\prs{e_{n - \ell + 1}, \ldots, e_n}$
אינו נמצא במרחב העמודות של
$\brs{T^n}_B$,
כיוון ש־%
$\ell$
השורות האחרונות במטריצה הן שורות אפסים.
לכן אף אחד מהוקטורים ב־%
$\Span \prs{v_{n - \ell + 1}, \ldots, v_n}$
אינו נמצא ב־%
$\im\prs{T^n}$
ונקבל כי
$\im\prs{T} \cap \ker\prs{T} = \set{0}$.
לכן הסכום
$\ker\prs{T^n} + \im\prs{T^n}$
הינו ישר.

ממשפט המימדים, מתקיים כי
\[\text{,} \dim\prs{V} = \dim\ker\prs{T^n} + \dim\im\prs{T^n}\]
אך כיוון שהמימד של סכום ישר הוא סכום המימדים מתקיים גם
\[\text{.} \dim\prs{\ker\prs{T^n} \oplus \im\prs{T^n}} = \dim \ker\prs{T^n} + \dim \im \prs{T^n}\]
לכן
$\ker\prs{T^n} \oplus \im\prs{T^n}$
תת־מרחב של
$V$
ממימד
$\dim\prs{V}$,
ולכן
\[\text{,} V = \ker\prs{T^n} \oplus \im\prs{T^n}\]
כנדרש.
\end{description}

\end{solution}

\newpage

\begin{question}
יהי
$\prs{V, \trs{\cdot, \cdot}}$
מרחב מכפלה פנימית סוף־מימדי מעל
$\mbb{C}$.
יהי
$T \in \End_{\mbb{C}}\prs{V}$
נילפוטנטי כך ש־%
$\trs{T\prs{v}, v} \in \mbb{R}$
לכל
$v \in V$.
האם בהכרח מתקיים ש־%
$T = 0$?
\end{question}

\begin{solution}
כן. נציג שתי דרכים להוכיח זאת.

\begin{description}
\item[דרך 1:]
מטענה מההרצאה, עבור אופרטור
$S$
על מרחב מכפלה פנימית מרוכב
$U$
מתקיים כי
$S^* = S$
אם ורק אם
$\trs{S\prs{u}, u} \in \mbb{R}$
לכל
$u \in U$.
נקבל מכך כי
$T^* = T$.
ובפרט כי
$T$
נורמלי.

ממשפט הפירוק הספקטרלי, לאופרטור נורמלי על מרכב מכפלה פנימית מרוכב קיים בסיס אורתונורמלי מלכסן.
לכן קיים
$B$
אורתונורמלי עבורו
$\brs{T}_B$
מטריצה אלכסונית.
אך
$T$
נילפוטנטי, וראינו כי מעל
$\mbb{C}$
זה שקול לכך ש־$0$ הינו ערך עצמי יחיד של
$T$.
לכן
$\brs{T}_B = 0$
ולכן
$T = 0$.

\item[דרך 2:]
נניח בשלילה כי
$T \neq 0$,
ויהי
$v \in V$
עבורו
$T\prs{v} \neq 0$
אך
$T^2\prs{v} = 0$.

יהיו
$\alpha, \beta \in \mbb{C}$
ונסמן
$u \coloneqq \alpha v + \beta T\prs{v}$.
אז
\begin{align*}
\trs{T\prs{u}, u} &=
\trs{\alpha T\prs{v} + \beta T^2\prs{v}, \alpha v + \beta T\prs{v}}
\\&=
\alpha^2 \trs{T\prs{v}, v} + \alpha \beta \trs{T\prs{v}, T\prs{v}}
\end{align*}
ביטוי ממשי מההנחה.
כיוון שלקחנו
$\alpha, \beta \in \mbb{C}$
כלליים, מתקיים שוויון גם עבור הבחירה
$\alpha = 1$,
ואז
\begin{align*}
\text{.} \trs{T\prs{u}, u} = \trs{T\prs{v}, v} + \beta \trs{T\prs{v}, T\prs{v}} = \trs{T\prs{v}, v} + \beta \norm{T\prs{v}}^2
\end{align*}
הביטוי
$\trs{T\prs{v}, v}$
גם הוא ממשי מההנחה, והנורמה
$\norm{T\prs{v}}$
הינה ממשית ושונה מאפס, לכן נקבל כי
$\beta$
גם הוא חייב להיות ממשי.
אך
$\beta$
איבר כללי ב־%
$\mbb{C}$,
לכן זאת סתירה.
\end{description}
\end{solution}

\newpage

\begin{question}
האם קיימים בסיסים אורתונורמליים
$\mcal{U}$
ו־%
$\mcal{V}$
של
$\mbb{R}^3$,
ואופרטור
$T \in \End_{\mbb{R}}\prs{\mbb{R}^3}$,
כך שהמטריצות
$\brs{T}_{\mcal{U}}$
ו־%
$\brs{T}_{\mcal{V}}$
אינן חופפות?
\end{question}

\begin{solution}
לא.

יהיו
$\mcal{U}, \mcal{V}$
בסיסים של
$\mbb{R}^3$
ויהי
$T \in \End_{\mbb{R}}\prs{\mbb{R}^3}$.
אז
\begin{align} \label{eq:change-of-basis}
\prs{M^{\mcal{U}}_{\mcal{V}}}^{-1} \brs{T}_{\mcal{U}} M^{\mcal{U}}_{\mcal{V}} = \brs{T}_{\mcal{V}}
\end{align}
כאשר
$M^{\mcal{U}}_{\mcal{V}}$
המטריצה היחידה
$A$
שמקיימת
$A \brs{v}_{\mcal{U}} = \brs{v}_{\mcal{V}}$
לכל
$v \in \mbb{R}^3$.

מתקיים
$M^{\mcal{U}}_{\mcal{V}} = M^{E}_{\mcal{V}} M^{\mcal{U}}_E$
לפי נוסחה למטריצות מעבר בסיס, וכן
$M^{E}_{\mcal{V}} = \prs{M^{\mcal{V}}_E}^{-1}$.
המטריצות
$U \coloneqq M^{\mcal{U}}_E$
ו־%
$V \coloneqq M^{\mcal{V}}_E$
הינן מטריצות עם עמודות שמהוות בסיסים אורתונורמליים, כי עמודותיהן הן איברי
$\mcal{U}$
ו־%
$\mcal{V}$
בהתאמה.
מטענה מההרצאה, מטריצה
$A \in \Mat_n\prs{\mbb{F}}$
הינה אורתוגונלית (או מעל $\mbb{C}$, אוניטרית) אם ורק אם עמודותיה מהוות בסיס אורתונורמלי ל־%
$\mbb{F}^n$,
ולכן
$U,V$
שתיהן אורתוגונליות.
אז
$V^{-1}$
אורתוגונלית כהופכית של מטריצה אורתוגונלית (מתקיים כי
$\prs{V^{-1}}^* = \prs{V^*}^* = V$
וזאת ההופכית של
$V^{-1}$)
ולכן גם
$M^{\mcal{U}}_{\mcal{V}} = V^{-1} U$
אורתוגונלית ככפל של מטריצות אורתוגונליות (אם
$A,B$
אורתוגונליות, מתקיים כי
$\prs{AB}^* = B^* A^* = B^{-1} A^{-1} = \prs{AB}^{-1}$
ולכן גם
$AB$
אורתוגונלית).

לכן
\[\prs{M^{\mcal{U}}_{\mcal{V}}}^{-1} = \prs{M^{\mcal{U}}_{\mcal{V}}}^* = \prs{M^{\mcal{U}}_{\mcal{V}}}^t\]
כאשר בשוויון השני השתמשנו בכך שעבור מטריצה ממשית
$A$
מתקיים
$A^* = A^t$.
נקבל מהצבה ב־%
\eqref{eq:change-of-basis}
כי
\begin{align*}
\prs{M^{\mcal{U}}_{\mcal{V}}}^{t} \brs{T}_{\mcal{U}} M^{\mcal{U}}_{\mcal{V}} = \brs{T}_{\mcal{V}}
\end{align*}
ולכן
$\brs{T}_{\mcal{U}}$
ו־%
$\brs{T}_{\mcal{V}}$
חופפות.
\end{solution}

\end{document}