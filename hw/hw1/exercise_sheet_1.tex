\documentclass[a4paper,10pt,twoside,openany]{article}

\usepackage[lang=hebrew]{maths}
\usepackage{hebrewdoc}
\usepackage{stylish}
\usepackage{lipsum}
\let\bs\blacksquare

\setlength{\parindent}{0pt}

%%%%%%%%%%%%
% Styling %
%%%%%%%%%%%%

\usepackage{enumitem}

%%%%%%%%%%%%%
% Counters  %
%%%%%%%%%%%%%

\setcounter{section}{1}     
            
%BIBLIOGRAPHY
\usepackage[
backend=biber,
style=alphabetic,
]{biblatex}
\addbibresource{bibliography.bib} %Imports bibliography file

%%%%%%%%%%
% Title  %
%%%%%%%%%%
\title{
אלגברה ב' - גיליון תרגילי בית 1 \\
מטריצות מייצגות ודטרמיננטה
\\
\vspace{1cm}
\large{תאריך הגשה: 14.04.2025}
}
\date{}

\begin{document}
\maketitle

\begin{exercise}
יהי
$V$
מרחב וקטורי מעל
$\mbb{R}$
ותהי
$T \colon V \to V$
לינארית המקיימת
$T^2 = -5 \id_V$.
\begin{enumerate}
\item הוכיחו כי לכל
$v \in V \setminus \set{0}$
הקבוצה
$\set{v, Tv}$
בלתי־תלויה לינארית.
\item נתון גם כי
$\dim V = 2$.
הוכיחו כי קיים בסיס בו
$T$
מיוצגת על ידי
$\pmat{0 & 1 \\ -5 & 0}$.
\end{enumerate}
\end{exercise}

\begin{exercise}
יהי
$B = \prs{v_i}_{i \in [n]}$
בסיס למרחב וקטורי
$V$.
נתונה
$T \colon V \to V$
הפיכה המקיימת
\[\text{.} T\prs{v_1 + 2 v_2} = \sum_{i \in [n]} v_i\]
מצאו את סכום איברי
$\brs{T^{-1}}_B$.
\end{exercise}

\begin{exercise}
חשבו את הדטרמיננטה של המטריצות המרוכבות הבאות.
\begin{align*}
A &= \pmat{1 & 2 & 3 \\ 4 & 5 & 6 \\ 7 & 8 & 9} \\
B &= \pmat{0 & 1 & 0 & 0 & 0 & 0 & 0
\\ 1 & 0 & 0 & 0 & 0 & 0 & 0 \\
0 & 0 & 0 & 0 & 0 & 1 & 0 \\
0 & 0 & 1 & 0 & 0 & 0 & 0 \\
0 & 0 & 0 & 1 & 0 & 0 & 0 \\
0 & 0 & 0 & 0 & 0 & 0 & 1 \\
0 & 0 & 0 & 0 & 1 & 0 & 0}\\
C &= \pmat{4 & 0 & 7 & 0 & 5 & 2 \\ 4 & 0 & -5 & 0 & 4 & -3 \\ 1 & 0 & 1 & 0 & 0 & 0 \\ -2 & 3 & -3 & 5 & 15 & -9 \\ -6 & 0 & 5 & 2 & 4 & -5 \\ 0 & 0 & 5 & 0 & 0 & 0}
\end{align*}
\end{exercise}

\newpage

\begin{exercise}
יהי
$p = \sum_{i = 0}^n a_i x^i \in \mbb{F}\brs{x}$
פולינום
\emph{מתוקן}
(כלומר,
$a_n = 1$).
ותהי
\[\text{.} C\prs{p} \ceq \pmat{0 & 0 & \cdots & 0 & -a_0 \\ 1 & 0 & \cdots & 0 & -a_1 \\ 0 & 1 & \cdots & 0 & -a_2 \\ \vdots & \vdots & \ddots & \vdots & \vdots \\ 0 & 0 & \dots & 1 & -a_{n-1}} \in \Mat_n\prs{\mbb{F}}\]
הראו כי
\[\text{.} \det\prs{xI - C\prs{p}} = p\prs{x}\]
\end{exercise}

\begin{exercise}
תהי
$T \colon V \to V$
העתקה לינארית ויהיו
$B$
בסיס של
$V$.
ראינו בתרגול שאפשר להגדיר
\begin{align*}
\det T &\ceq \det\prs{\brs{T}_B}
\end{align*}
ושההגדרה אינה תלויה בבחירת הבסיס.

יהי
$V = \Mat_2\prs{\mbb{C}}$
ותהי
\begin{align*}
T \colon V &\to V \\
\text{.} \hphantom{lala} A &\mapsto \pmat{0 & 1 \\ 1 & 0} A \pmat{0 & 1 \\ 1 & 0}
\end{align*}
חשבו את
$\det\prs{T}$.
\end{exercise}

\begin{exercise}
תהי
$A \in \Mat_n\prs{\mbb{R}}$
בלי ערכים עצמיים ממשיים.
הראו כי
$\det\prs{A} > 0$.
\end{exercise}

\end{document}